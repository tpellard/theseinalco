\documentclass[a4paper,11pt]{book}
\usepackage{theseinalco}
\usepackage{setspace}
\doublespacing
% \onehalfspacing
\usepackage{csquotes}
\usepackage{booktabs}

\usepackage{polyglossia}
\setmainlanguage{french}\frenchspacing
\setmainfont[Mapping=tex-text,Ligatures=Common,Numbers=OldStyle,SmallCapsFeatures={WordSpace=2,LetterSpace=8}]{Linux Libertine O}
\usepackage{microtype}

\title{La connaissance des savoirs dans les mondes antiques et modernes}
\subtitle{Regards croisés sur les centres et les périphéries}
\author{Victoire Thésard}
\preauthor{présentée par}
\logo{\includegraphics[scale=.8]{inalco-logo-color}\hfill\includegraphics[scale=.8]{inalco-logo-color}}
\universite{institut national des langues et civilisations orientales}
\ecoledoctorale{École doctorale n\textsuperscript{o}265:\quad \textit{Langues, Littératures et Sociétés du monde}}
\unite{Langues et civilisations (\textsc{umr} 0000)}
\diplome{Thèse}

% En cas de co-tutelle
\diplome{Thèse en cotutelle}
\couniversite{Université orientale des savoirs}
\coecoledoctorale{École doctorale n\textsuperscript{o}xxx5:\quad \textit{Mondes et cultures}}
\counite{Savoirs du monde}

\date{soutenance prévue le 31 décembre 20xx}%Soutenue publiquement le
\discipline{discipline: Langues et civilisations}
\grade{pour obtenir le grade de docteur de l'\textsc{inalco}}
\direction{Thèse dirigée par:}
\directeur{M.}{André}{Nonyme}{Professeur, Université de la Science}
\rapporteur{M\textsuperscript{me}}{Marie-Pierre}{Machinchose}{Directrice de recherches, Centre de recherches sur le Savoir}
\rapporteur{M.}{Pierre}{Pauljacques}{Professeur, Université de la Science}

\jury{M\textsuperscript{me}}{Marie-Pierre}{Machinchose}{Directrice de recherches, Centre de recherches sur le Savoir}
\jury{M.}{Pierre}{Pauljacques}{Professeur, Université de la Science}
\jury{M.}{Paul}{Pierrejacques}{Chargé de recherches, Centre de recherches sur le Savoir}
\jury{M\textsuperscript{me}}{Jacqueline}{Pierrepaul}{Directrice d'études, École des hautes études sur le Monde et Institut international des connaissances}

\usepackage{kantlipsum}

\begin{document}
\maketitle

\frontmatter
\chapter{Remerciements}
\kant%[1]

\mainmatter
\part{Ma partie}
\chapter{Mon chapitre}

\section{Section}
Ceci est un \enquote{test}.\footnote{\kant[1]}
\kant[1-2]
\begin{itemize}
 \item liste
 \item liste
\end{itemize}

\begin{figure}
 \centering
 \fbox{Test}
 \caption{Ceci est une figure}
\end{figure}

\begin{table}[htb]
\centering
\caption{Un joli tableau}\label{tab:joli}
\begin{tabular}{ccc}\toprule
A & B & C\\\midrule
élément & case & texte\\
case & élément & texte \\
texte & case & élément \\
\bottomrule\end{tabular}
\end{table}

\part{Mon autre partie}

\chapter{Mon titre de chapitre est long voire extrêmement long}
\section{Section}
\subsection{Sous-section}
\subsubsection{Sous-sous-section}
\kant
% \clearpage
\begin{quote}
\kant[1]
\end{quote}

\kant[1]

\newenvironment{macitation}{\begin{quote}\singlespacing\small}{\end{quote}}

\begin{macitation}
\kant[1] texte texte\footnote{Note}
\end{macitation}

\kant[1]

\section{Section}
\subsection{Sous-section}
% \kant[5]
\subsubsection{Sous-sous-section}
\kant[5]

\chapter{Autre chapitre}
\section{Section}
\subsection{Sous-section}
\subsubsection{Sous-sous-section}
\kant

\appendix
\chapter{Une annexe}
\kant

\backmatter

\tableofcontents
\listoffigures
\listoftables

\dernierepage
\section*{Résumé}
\kant[1]
\paragraph{Mots-clés} mot, mot, mot

\section*{Abstract}
\kant[1]
\paragraph{Keywords} word, word, word
\end{document}